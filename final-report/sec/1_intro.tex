\section{Introduction}
\label{sec:intro}

Image compression plays a vital role in modern visual communication systems, enabling efficient storage and transmission of large volumes of image data. From digital photography and medical imaging to web services and embedded systems, effective compression algorithms are crucial for managing bandwidth and storage constraints. Especially with the dawn of machine learning and in particular image classification, the need for reducing image sizes while retaining the general perceptual quality, has become even more apparent.

In modern visual systems, image compression enables efficient handling of visual data through two fundamental approaches. \textbf{Lossy compression} techniques like JPEG achieve significant size reduction by strategically discarding perceptually redundant information through its multi-stage pipeline of color space conversion (RGB to YCbCr), chrominance subsampling, block-based DCT transformation and quantization, and (in itself lossless) entropy coding \cite{haines1992compression,jpegOverview2025}. Modern successors like WebP and AVIF build upon these principles with enhanced predictive coding. In contrast, \textbf{lossless compression} methods such as PNG and FLIF preserve every bit of original data through DEFLATE compression and advanced context modeling respectively \cite{flif2016}, making them indispensable for medical imaging or archival purposes where fidelity is paramount. While JPEG's fixed configuration limits customization, newer but unfortunately still unpopular formats like JPEG XL \cite{jpegxl2021} can be seen as versatile alternatives that unify both lossy and lossless approaches while adding features like HDR support and progressive decoding.

\noindent In this project, we address the limitations of standard JPEG compression - particularly its rigid configuration that restricts parameter experimentation - by developing a \textbf{modular, Python-based image compression framework}. Inspired by JPEG's core architecture but designed for academic exploration, in particular regarding image classification, our system implements the traditional transform coding pipeline while exposing key variables for analysis, such as block size, quantization strategy, and downsampling rate. 

We further extended the project by incorporating parameter sweeps and classification-based evaluation, enabling both qualitative and quantitative assessments of compression performance. The primary goals of this work were to (1) build a working JPEG-like compression and decompression system from the ground up, (2) understand and visualize the effects of different parameter choices on output quality and file size, and (3) test whether compressed images preserve semantic content in machine-learning-based classification tasks. Our system design was partially inspired by recent advances in semantic-aware image compression~\cite{semanticDiffusion2023}.

%Regarding the mathematical notation, we use bold upper-case letters to denote arrays \(\bm{A}\), bold lower-case letters to denote vectors \(\bm{v}\), whereas scalars as plain \(s\).

The remainder of this paper is organized as follows: Section 2 describes our system implementation, Section 3 presents experimental results and analysis, and Section 4 offers concluding remarks.
The full source code and configuration files for this system are available at: \url{https://github.com/Roude/EECE5698-image-compression-roud-drag-elar}.
